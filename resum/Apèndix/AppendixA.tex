% !TEX root = ../main.tex

%----------------------------------------------------------------------------------------
% APPENDIX A
%----------------------------------------------------------------------------------------

\chapter{Fitxes completes de Casos d'Ús}
\label{app:casos_us}

A continuació es detallen els casos d'ús principals del sistema, descrivint la interacció entre els actors i els diferents microserveis.

\section{UC-01: Registrar-se}
\begin{itemize}
    \item \textbf{Descripció}: L'usuari crea un compte nou.
    \item \textbf{Actors}: Usuari.
    \item \textbf{Precondicions}: No haver iniciat sessió.
    \item \textbf{Postcondicions}: Compte creat i sessió iniciada.
    \item \textbf{Escenari principal}:
    \begin{enumerate}
        \item El client envia una sol·licitud de registre al servei \textbf{UserAuthentication} a través del Gateway.
        \item El servei valida les dades rebudes:
        \begin{itemize}
            \item Comprova que el format del nom d'usuari, contrasenya i email siguin correctes.
            \item Verifica que el nom d'usuari no estigui ja en ús.
            \item Consulta a \textbf{UserManagement} per assegurar que l'email no estigui registrat.
        \end{itemize}
        \item Si les validacions són correctes, \textbf{UserAuthentication} guarda les credencials (nom d'usuari i contrasenya xifrada) i inicia la creació de dades en altres serveis:
        \begin{itemize}
            \item Fa una crida a \textbf{UserManagement} per guardar la informació personal de l'usuari (email, nom i cognoms).
            \item Crida a \textbf{FileManagement} per crear la carpeta arrel de l'usuari.
            \item Finalment, envia un missatge asíncron a través de RabbitMQ a \textbf{SyncService} per generar l'estat inicial de sincronització (\emph{snapshot}) amb la carpeta arrel.
        \end{itemize}
        \item Un cop finalitzat el procés, \textbf{UserAuthentication} retorna els tokens d'accés i de refresc per iniciar la sessió automàticament.
    \end{enumerate}
\end{itemize}

\section{UC-02: Iniciar sessió}
\begin{itemize}
    \item \textbf{Descripció}: L'usuari inicia sessió amb el seu nom i contrasenya.
    \item \textbf{Actors}: Usuari.
    \item \textbf{Precondicions}: Tenir un compte vàlid.
    \item \textbf{Postcondicions}: Sessió activa.
    \item \textbf{Escenari principal}:
    \begin{enumerate}
        \item El client envia el nom d'usuari i la contrasenya al servei \textbf{UserAuthentication}.
        \item El servei busca l'usuari a la base de dades a partir del seu nom.
        \item Si l'usuari existeix, compara la contrasenya rebuda amb la versió xifrada emmagatzemada.
        \item Si les credencials són correctes, genera un nou token d'accés (JWT) de curta durada i un token de refresc de llarga durada.
        \item Finalment, retorna els dos tokens al client per iniciar la sessió i mantenir-la activa.
    \end{enumerate}
\end{itemize}

\section{UC-03: Actualitzar perfil}
\begin{itemize}
    \item \textbf{Descripció}: L'usuari modifica les seves dades personals (nom, cognoms i email).
    \item \textbf{Actors}: Usuari.
    \item \textbf{Precondicions}: Sessió iniciada.
    \item \textbf{Postcondicions}: Dades de perfil actualitzades a la base de dades.
    \item \textbf{Escenari principal}:
    \begin{enumerate}
        \item El Gateway valida el token JWT i reenvia la petició a \textbf{UserManagement} amb l'ID de l'usuari.
        \item El servei busca l'usuari a la base de dades.
        \item Valida que el nou email no estigui en ús per un altre usuari.
        \item Si la validació és correcta, actualitza el nom, cognoms i email de l'usuari.
        \item Retorna les dades actualitzades.
    \end{enumerate}
\end{itemize}

\section{UC-04: Cercar usuaris}
\begin{itemize}
    \item \textbf{Descripció}: Permet localitzar usuaris per nom d'usuari o correu electrònic.
    \item \textbf{Actors}: Usuari.
    \item \textbf{Precondicions}: Sessió iniciada.
    \item \textbf{Postcondicions}: Retorna una llista d'usuaris que coincideixen amb el criteri de cerca.
    \item \textbf{Escenari principal}:
    \begin{enumerate}
        \item El Gateway valida el JWT i envia la petició de cerca a \textbf{UserManagement}.
        \item \textbf{UserManagement} fa una crida a \textbf{UserAuthentication} per buscar coincidències per nom d'usuari.
        \item Paral·lelament, \textbf{UserManagement} busca a la seva pròpia base de dades coincidències per email.
        \item Combina i retorna una llista d'usuaris (nom d'usuari i email) que compleixen el criteri de cerca.
    \end{enumerate}
\end{itemize}

\section{UC-05: Llistar usuaris (administració)}
\begin{itemize}
    \item \textbf{Descripció}: Un administrador o superadministrador consulta la llista completa d'usuaris del sistema.
    \item \textbf{Actors}: Administrador, Superadministrador.
    \item \textbf{Precondicions}: Rol d'administrador o superadministrador.
    \item \textbf{Postcondicions}: Llista de tots els usuaris amb les seves dades.
    \item \textbf{Escenari principal}:
    \begin{enumerate}
        \item El Gateway valida el JWT i el rol del sol·licitant, i reenvia la petició a \textbf{UserManagement}.
        \item \textbf{UserManagement} obté la llista de tots els usuaris de la seva base de dades.
        \item A continuació, fa una crida a \textbf{UserAuthentication} per obtenir les dades d'autenticació (nom d'usuari i rol) de cada usuari.
        \item Si el sol·licitant és Superadministrador, la informació retornada per \textbf{UserAuthentication} inclou també la contrasenya.
        \item Finalment, \textbf{UserManagement} combina la informació i retorna una llista completa amb els detalls de cada usuari.
    \end{enumerate}
\end{itemize}

\section{UC-06: Actualitzar usuari (administració)}
\begin{itemize}
    \item \textbf{Descripció}: Un administrador o superadministrador modifica les dades d'un altre usuari, incloent el seu rol o contrasenya.
    \item \textbf{Actors}: Administrador, Superadministrador.
    \item \textbf{Precondicions}: Rol administratiu i respectar la jerarquia de permisos.
    \item \textbf{Postcondicions}: Dades de l'usuari objectiu actualitzades.
    \item \textbf{Escenari principal}:
    \begin{enumerate}
        \item El Gateway valida el JWT i el rol, i envia la petició a \textbf{UserManagement}.
        \item \textbf{UserManagement} fa una crida a \textbf{UserAuthentication} per comprovar els rols tant del sol·licitant com de l'usuari a modificar.
        \item Es valida la jerarquia: un Administrador només pot modificar usuaris amb rol \texttt{USER}, mentre que un Superadministrador pot modificar qualsevol usuari excepte treure's a si mateix el rol de Superadministrador.
        \item \textbf{UserManagement} actualitza la informació personal (nom, email) a la seva base de dades.
        \item Simultàniament, fa una crida a \textbf{UserAuthentication} perquè actualitzi les dades d'autenticació (nom d'usuari, rol i, si s'escau, la contrasenya).
    \end{enumerate}
\end{itemize}

\section{UC-07: Eliminar usuari (administració)}
\begin{itemize}
    \item \textbf{Descripció}: Un administrador o superadministrador inicia el procés d'eliminació completa d'un usuari.
    \item \textbf{Actors}: Administrador, Superadministrador.
    \item \textbf{Precondicions}: Rol administratiu i respectar la jerarquia de permisos.
    \item \textbf{Postcondicions}: S'inicia l'eliminació asíncrona de les dades de l'usuari a tots els serveis.
    \item \textbf{Escenari principal}:
    \begin{enumerate}
        \item El Gateway valida el JWT i el rol, i reenvia la petició a \textbf{UserManagement}.
        \item \textbf{UserManagement} crida a \textbf{UserAuthentication} per verificar la jerarquia de rols (un Administrador només pot eliminar usuaris \texttt{USER}, un Superadministrador pot eliminar Administradors).
        \item Si es compleixen els permisos, \textbf{UserManagement} envia un missatge d'eliminació a una cua de RabbitMQ.
        \item Aquest missatge és rebut per \textbf{UserAuthentication}, que esborra les credencials de l'usuari i, al seu torn, envia un nou missatge de notificació d'eliminació.
        \item Aquest segon missatge és consumit per la resta de microserveis (\textbf{FileManagement}, \textbf{FileSharing}, etc.), que procedeixen a eliminar de forma asíncrona totes les dades associades a l'usuari.
    \end{enumerate}
\end{itemize}

\section{UC-08: Crear o pujar arxius}
\begin{itemize}
  \item \textbf{Descripció}: Pujada de fitxers o creació de carpetes.
  \item \textbf{Actors}: Usuari.
  \item \textbf{Precondicions}: Permís d'escriptura a la carpeta de destinació.
  \item \textbf{Postcondicions}: Nou element emmagatzemat i notificat.
  \item \textbf{Escenari principal}:
    \begin{enumerate}
        \item El Gateway valida el token JWT i reenvia la petició a \textbf{FileManagement}.
        \item El servei consulta a \textbf{FileAccessControl} per verificar que l'usuari té permís d'escriptura (\texttt{WRITE}) a la carpeta de destinació.
        \item Si el permís és correcte, crea les metadades de l'arxiu (nom, mida, etc.) a la seva base de dades.
        \item Emmagatzema el contingut del fitxer al sistema d'arxius del servidor, utilitzant un ID únic com a nom.
        \item Sol·licita a \textbf{FileAccessControl} que assigni el permís de propietari (\texttt{ADMIN}) sobre el nou element a l'usuari que l'ha pujat.
        \item Finalment, envia un missatge asíncron a \textbf{SyncService} per notificar la creació i actualitzar els clients.
    \end{enumerate}
\end{itemize}

\section{UC-09A: Renomenar un element}
\begin{itemize}
    \item \textbf{Descripció}: Canvia el nom d'un arxiu o carpeta existent.
    \item \textbf{Actors}: Usuari.
    \item \textbf{Precondicions}: Permís d'escriptura sobre l'element.
    \item \textbf{Postcondicions}: L'element té el nou nom assignat.
    \item \textbf{Escenari principal}:
    \begin{enumerate}
        \item El Gateway valida el JWT i envia la sol·licitud a \textbf{FileManagement} amb el nou nom.
        \item \textbf{FileManagement} crida a \textbf{FileAccessControl} per verificar que l'usuari té permís d'escriptura (\texttt{WRITE}) sobre l'element.
        \item Actualitza el nom de l'element a la base de dades.
        \item Envia un missatge d'actualització (\texttt{update}) a \textbf{SyncService} per notificar el canvi als clients.
    \end{enumerate}
\end{itemize}

\section{UC-09B: Moure un element}
\begin{itemize}
    \item \textbf{Descripció}: Canvia la ubicació d'un arxiu o carpeta a una nova carpeta de destinació.
    \item \textbf{Actors}: Usuari.
    \item \textbf{Precondicions}: Permís d'escriptura sobre l'element a moure i sobre la carpeta de destinació.
    \item \textbf{Postcondicions}: L'element es troba a la nova ubicació.
    \item \textbf{Escenari principal}:
    \begin{enumerate}
        \item El Gateway valida el JWT i envia la sol·licitud a \textbf{FileManagement} amb l'ID de l'element i de la destinació.
        \item \textbf{FileManagement} verifica a \textbf{FileAccessControl} el permís d'escriptura (\texttt{WRITE}) sobre l'origen i la destinació.
        \item Valida que l'operació no generi una dependència circular (p. ex., moure una carpeta dins d'ella mateixa).
        \item Actualitza la referència a la carpeta pare de l'element a la base de dades.
        \item Envia un missatge d'actualització (\texttt{update}) a \textbf{SyncService} per notificar el canvi als clients.
    \end{enumerate}
\end{itemize}

\section{UC-09C: Copiar un element}
\begin{itemize}
    \item \textbf{Descripció}: Crea un duplicat d'un arxiu o carpeta en una ubicació de destinació.
    \item \textbf{Actors}: Usuari.
    \item \textbf{Precondicions}: Permís de lectura sobre l'element a copiar i d'escriptura sobre la carpeta de destinació.
    \item \textbf{Postcondicions}: Es crea una còpia de l'element a la destinació.
    \item \textbf{Escenari principal}:
    \begin{enumerate}
        \item El Gateway valida el JWT i envia la sol·licitud a \textbf{FileManagement}.
        \item \textbf{FileManagement} verifica a \textbf{FileAccessControl} el permís de lectura (\texttt{READ}) sobre l'origen i d'escriptura (\texttt{WRITE}) sobre la destinació.
        \item Crea les noves entitats a la base de dades per a la còpia, assignant nous IDs.
        \item Si és un fitxer, duplica el contingut físic al sistema d'emmagatzematge.
        \item Sol·licita a \textbf{FileAccessControl} que assigni el permís de propietari (\texttt{ADMIN}) sobre el nou element a l'usuari.
        \item Envia un missatge de creació (\texttt{create}) a \textbf{SyncService} per notificar el nou element als clients.
    \end{enumerate}
\end{itemize}

\section{UC-10: Descarregar}
\begin{itemize}
  \item \textbf{Descripció}: Obtenir el contingut d'un arxiu o carpeta.
  \item \textbf{Actors}: Usuari.
  \item \textbf{Precondicions}: Permís de lectura sobre l'element sol·licitat.
  \item \textbf{Postcondicions}: Arxiu descarregat pel client.
  \item \textbf{Escenari principal}:
    \begin{enumerate}
        \item El Gateway valida el JWT i reenvia la petició a \textbf{FileManagement}.
        \item Aquest consulta \textbf{FileAccessControl} per confirmar que l'usuari té permís de lectura (\texttt{READ}).
        \item Si els permisos són correctes, recupera el fitxer del sistema d'emmagatzematge. Si és una carpeta, la comprimeix en format ZIP.
        \item Retorna el contingut com un flux de dades (\emph{stream}) perquè el client iniciï la descàrrega.
    \end{enumerate}
\end{itemize}

\section{UC-11: Enviar a la paperera}
\begin{itemize}
  \item \textbf{Descripció}: Moure elements a la paperera.
  \item \textbf{Actors}: Usuari.
  \item \textbf{Precondicions}: Permís d'escriptura sobre l'element.
  \item \textbf{Postcondicions}: Element marcat com a eliminat i visible a la paperera.
  \item \textbf{Escenari principal}:
      \begin{enumerate}
        \item El Gateway valida el JWT i envia la petició a \textbf{TrashService}.
        \item \textbf{TrashService} verifica a \textbf{FileAccessControl} que l'usuari té permís d'escriptura.
        \item Crida a \textbf{FileManagement} perquè marqui l'element i els seus descendents com a eliminats (sense esborrar-los físicament).
        \item Crea un registre a la seva pròpia base de dades (\texttt{TrashRecord}) per cada element mogut, emmagatzemant la data d'eliminació i de caducitat.
        \item \textbf{FileManagement} notifica a \textbf{SyncService} el canvi d'estat per actualitzar els clients.
    \end{enumerate}
\end{itemize}

\section{UC-12: Eliminar permanentment}
\begin{itemize}
  \item \textbf{Descripció}: Esborrar definitivament un element de la paperera.
  \item \textbf{Actors}: Usuari.
  \item \textbf{Precondicions}: Element a la paperera i ser-ne el propietari.
  \item \textbf{Postcondicions}: Element eliminat de forma permanent.
  \item \textbf{Escenari principal}:
    \begin{enumerate}
        \item El Gateway envia la petició a \textbf{TrashService}.
        \item El servei verifica que l'usuari és el propietari de l'element.
        \item Crida a \textbf{FileManagement} per esborrar l'arxiu físic i les seves metadades.
        \item Canvia l'estat del \texttt{TrashRecord} a \texttt{PENDING\_DELETION} i inicia un procés de purga asíncron.
        \item A través de missatges per cua, ordena a \textbf{FileAccessControl} i \textbf{FileSharing} que eliminin totes les regles associades a l'element.
        \item Un cop confirmat per tots els serveis, el \texttt{TrashRecord} s'esborra.
        \item Es notifica a \textbf{SyncService} per eliminar les còpies locals de l'element.
    \end{enumerate}
\end{itemize}

\section{UC-13: Compartir arxius}
\begin{itemize}
  \item \textbf{Descripció}: Concedir o revocar accessos sobre elements a altres usuaris.
  \item \textbf{Actors}: Usuari.
  \item \textbf{Precondicions}: Permís de propietari o d'administrador sobre l'element.
  \item \textbf{Postcondicions}: Els usuaris seleccionats obtenen o perden l'accés indicat.
  \item \textbf{Escenari principal}:
    \begin{enumerate}
        \item El Gateway valida el JWT i passa la petició a \textbf{FileSharing}.
        \item El servei verifica a \textbf{FileAccessControl} que el sol·licitant és el propietari (\texttt{ADMIN}) de l'element.
        \item Consulta a \textbf{UserManagement} per obtenir l'ID de l'usuari amb qui es vol compartir.
        \item Sol·licita a \textbf{FileAccessControl} que creï una nova regla d'accés (lectura o escriptura) per a l'usuari convidat sobre l'element i els seus descendents (si es una carpeta).
        \item Desa un registre de la compartició a la seva base de dades.
        \item Notifica a \textbf{SyncService} a través de RabbitMQ per propagar els canvis als clients implicats.
    \end{enumerate}
\end{itemize}

\section{UC-13A: Revocar accés a un arxiu (propietari)}
\begin{itemize}
    \item \textbf{Descripció}: El propietari d'un element compartit revoca el permís d'accés a un altre usuari.
    \item \textbf{Actors}: Usuari (propietari).
    \item \textbf{Precondicions}: Ser el propietari (\texttt{ADMIN}) de l'element.
    \item \textbf{Postcondicions}: L'usuari objectiu perd l'accés a l'element.
    \item \textbf{Escenari principal}:
    \begin{enumerate}
        \item El Gateway valida el JWT i reenvia la petició a \textbf{FileSharing} amb l'ID de l'element i el nom de l'usuari a qui es revoca el permís.
        \item \textbf{FileSharing} crida a \textbf{FileAccessControl} per verificar que el sol·licitant és el propietari (\texttt{ADMIN}) de l'element.
        \item Sol·licita a \textbf{UserManagement} l'ID de l'usuari a eliminar.
        \item Crida a \textbf{FileAccessControl} per eliminar la regla d'accés associada a l'usuari i a l'element (i als seus descendents, si és una carpeta).
        \item Elimina el registre corresponent de la taula de \texttt{SharedAccess} de la seva pròpia base de dades.
        \item Notifica a \textbf{SyncService} a través de RabbitMQ per actualitzar els clients implicats.
    \end{enumerate}
\end{itemize}

\section{UC-13B: Deixar de seguir un arxiu compartit (receptor)}
\begin{itemize}
    \item \textbf{Descripció}: Un usuari elimina un element que algú altre havia compartit amb ell.
    \item \textbf{Actors}: Usuari (receptor).
    \item \textbf{Precondicions}: Un altre usuari ha compartit un element amb l'usuari actual.
    \item \textbf{Postcondicions}: L'element desapareix de la llista d'arxius compartits de l'usuari.
    \item \textbf{Escenari principal}:
    \begin{enumerate}
        \item El Gateway valida el JWT i reenvia la petició a \textbf{FileSharing}. El flux és idèntic a \texttt{UC-13A}, però en aquest cas el sol·licitant és el mateix usuari a qui es revocarà el permís.
        \item \textbf{FileSharing} comprova que el sol·licitant i l'usuari a eliminar són el mateix.
        \item Es crida a \textbf{FileAccessControl} per eliminar la regla d'accés i a la base de dades de \textbf{FileSharing} per eliminar el registre de compartició.
        \item Finalment, es notifica a \textbf{SyncService} per actualitzar la interfície de l'usuari.
    \end{enumerate}
\end{itemize}

\section{UC-14: Actualització en temps real}
\begin{itemize}
    \item \textbf{Descripció}: Manté els clients sincronitzats amb els canvis del sistema de fitxers mitjançant WebSockets.
    \item \textbf{Actors}: Usuari.
    \item \textbf{Precondicions}: Sessió iniciada.
    \item \textbf{Postcondicions}: El client rep esdeveniments de sincronització i actualitza el seu estat local.
    \item \textbf{Escenari principal}:
    \begin{enumerate}
        \item El client (Tauri o web) estableix una connexió WebSocket amb el Gateway, enviant el token JWT a les capçaleres.
        \item El Gateway valida el token i reenvia la connexió a \textbf{SyncService}, afegint a les capçaleres l'ID d'usuari i un ID de connexió únic.
        \item \textbf{SyncService} registra la connexió WebSocket associada a l'usuari. Si el client és d'escriptori (Tauri), li envia l'estat actual complet del seu arbre de fitxers (\textit{snapshot}).
        \item Quan un altre servei (com \textbf{FileManagement} o \textbf{FileSharing}) realitza una acció que modifica l'estructura de fitxers, envia un missatge a una cua de RabbitMQ.
        \item \textbf{SyncService} consumeix aquest missatge, actualitza l'snapshot de l'usuari afectat i propaga el canvi.
        \item Per als clients d'escriptori, envia el \textit{snapshot} actualitzat. Per als clients web, envia una ordre de refresc (\texttt{updated\_tree}) per a la part de la interfície afectada. Això es fa a través de la connexió WebSocket corresponent.
    \end{enumerate}
\end{itemize}

\section{UC-15: Canviar contrasenya}
\begin{itemize}
    \item \textbf{Descripció}: L'usuari actualitza la seva contrasenya.
    \item \textbf{Actors}: Usuari.
    \item \textbf{Precondicions}: Sessió iniciada i coneixement de la contrasenya actual.
    \item \textbf{Postcondicions}: Contrasenya modificada a la base de dades.
    \item \textbf{Escenari principal}:
    \begin{enumerate}
        \item El Gateway valida el JWT i reenvia la petició a \textbf{UserAuthentication} amb l'ID de l'usuari i les contrasenyes (antiga i nova).
        \item El servei recupera l'usuari de la base de dades.
        \item Compara la contrasenya antiga rebuda amb el que hi ha emmagatzemat per verificar-la.
        \item Valida que la nova contrasenya compleix els requisits de seguretat (llargada, majúscules, minúscules i números).
        \item Si tot és correcte, xifra la nova contrasenya i l'actualitza a la base de dades.
    \end{enumerate}
\end{itemize}

\section{UC-16: Eliminar compte}
\begin{itemize}
    \item \textbf{Descripció}: L'usuari sol·licita esborrar definitivament el seu propi compte i tots els seus arxius.
    \item \textbf{Actors}: Usuari.
    \item \textbf{Precondicions}: Sessió iniciada.
    \item \textbf{Postcondicions}: S'inicia el procés d'eliminació asíncrona de totes les dades de l'usuari.
    \item \textbf{Escenari principal}:
    \begin{enumerate}
        \item El Gateway valida el JWT i passa la petició a \textbf{UserAuthentication}.
        \item \textbf{UserAuthentication} esborra l'entitat de l'usuari de la seva base de dades.
        \item A continuació, envia un missatge a una cua de RabbitMQ per notificar que un usuari ha estat eliminat.
        \item La resta de microserveis (\textbf{UserManagement}, \textbf{FileManagement}, etc.) estan subscrits a aquesta cua i, en rebre el missatge, cadascun inicia la purga de totes les dades associades a l'ID d'aquell usuari.
    \end{enumerate}
\end{itemize}

\section{UC-17: Restaurar des de la paperera}
\begin{itemize}
    \item \textbf{Descripció}: Recuperar un element prèviament enviat a la paperera.
    \item \textbf{Actors}: Usuari.
    \item \textbf{Precondicions}: L'element es troba a la paperera i l'usuari n'és el propietari.
    \item \textbf{Postcondicions}: Element restaurat a la seva ubicació original i ja no és visible a la paperera.
    \item \textbf{Escenari principal}:
    \begin{enumerate}
        \item El Gateway valida el JWT i envia la sol·licitud a \textbf{TrashService}.
        \item \textbf{TrashService} verifica que l'usuari que fa la petició és el propietari.
        \item Crida a \textbf{FileManagement} per canviar l'estat de l'element (i els seus descendents, si és una carpeta) a no eliminat.
        \item \textbf{TrashService} elimina de la seva base de dades tots els registres (\texttt{TrashRecord}) associats als elements restaurats.
        \item Finalment, \textbf{FileManagement} notifica a \textbf{SyncService} el canvi d'estat a través de RabbitMQ per actualitzar la vista dels clients.
    \end{enumerate}
\end{itemize}

\section{UC-18: Modificar contrasenya d'un altre usuari (Superadministració)}
\begin{itemize}
    \item \textbf{Descripció}: El Superadministrador canvia la contrasenya d'un altre usuari.
    \item \textbf{Actors}: Superadministrador.
    \item \textbf{Precondicions}: Rol de Superadministrador.
    \item \textbf{Postcondicions}: Contrasenya de l'usuari objectiu modificada.
    \item \textbf{Escenari principal}:
    \begin{enumerate}
        \item La petició arriba a \textbf{UserManagement} a través del Gateway.
        \item \textbf{UserManagement} crida a \textbf{UserAuthentication} per validar que el sol·licitant és Superadministrador.
        \item Si la validació és correcta, \textbf{UserManagement} envia una ordre d'actualització a \textbf{UserAuthentication} amb la nova contrasenya.
        \item \textbf{UserAuthentication} valida el format de la nova contrasenya, la xifra i l'emmagatzema a la seva base de dades.
    \end{enumerate}
\end{itemize}

\section{UC-19: Modificar un administrador (Superadministració)}
\begin{itemize}
    \item \textbf{Descripció}: El Superadministrador modifica les dades d'un usuari amb rol d'Administrador.
    \item \textbf{Actors}: Superadministrador.
    \item \textbf{Precondicions}: Rol de Superadministrador.
    \item \textbf{Postcondicions}: Dades de l'administrador actualitzades.
    \item \textbf{Escenari principal}:
    \begin{enumerate}
        \item Similar a \texttt{UC-06}, el Gateway reenvia la petició a \textbf{UserManagement}.
        \item \textbf{UserManagement} crida a \textbf{UserAuthentication} per verificar que el sol·licitant és \texttt{SUPER\_ADMIN} i l'usuari a modificar és \texttt{ADMIN}.
        \item Si la jerarquia de permisos és correcta, \textbf{UserManagement} actualitza la informació personal.
        \item Paral·lelament, \textbf{UserAuthentication} actualitza les dades d'autenticació (nom d'usuari, rol).
    \end{enumerate}
\end{itemize}

\section{UC-20: Eliminar un administrador (Superadministració)}
\begin{itemize}
    \item \textbf{Descripció}: El Superadministrador elimina un compte d'usuari amb rol d'Administrador.
    \item \textbf{Actors}: Superadministrador.
    \item \textbf{Precondicions}: Rol de Superadministrador.
    \item \textbf{Postcondicions}: S'inicia l'eliminació asíncrona de les dades de l'administrador.
    \item \textbf{Escenari principal}:
    \begin{enumerate}
        \item El Gateway valida el rol i reenvia la petició a \textbf{UserManagement}.
        \item \textbf{UserManagement} crida a \textbf{UserAuthentication} per verificar que el sol·licitant és \texttt{SUPER\_ADMIN} i l'usuari a eliminar és \texttt{ADMIN}.
        \item Si els permisos són correctes, s'inicia el mateix procés d'eliminació asíncrona descrit a \texttt{UC-07}, enviant missatges a la resta de serveis per purgar totes les dades associades.
    \end{enumerate}
\end{itemize}

\section{UC-21: Modificar nivell de permisos d'un usuari (Superadministració)}
\begin{itemize}
    \item \textbf{Descripció}: El Superadministrador canvia el rol d'un usuari (p. ex., de \texttt{USER} a \texttt{ADMIN}).
    \item \textbf{Actors}: Superadministrador.
    \item \textbf{Precondicions}: Rol de Superadministrador.
    \item \textbf{Postcondicions}: El rol de l'usuari objectiu és modificat.
    \item \textbf{Escenari principal}:
    \begin{enumerate}
        \item El flux és una variant de \texttt{UC-06}. La petició arriba a \textbf{UserManagement}.
        \item Es verifica a \textbf{UserAuthentication} que el sol·licitant té permisos per realitzar el canvi de rol. Un Superadministrador no pot rebaixar el seu propi rol.
        \item \textbf{UserManagement} envia la petició a \textbf{UserAuthentication} per actualitzar el camp \texttt{role} de l'usuari a la base de dades d'autenticació.
    \end{enumerate}
\end{itemize}

\section{UC-22: Sincronitzar arxius compartits}
\begin{itemize}
    \item \textbf{Descripció}: Manté els clients sincronitzats amb els canvis en arxius i carpetes que altres usuaris han compartit amb ells. Aquesta funcionalitat no va ser completament implementada.
    \item \textbf{Actors}: Usuari.
    \item \textbf{Precondicions}: Altres usuaris han compartit elements amb l'usuari actual.
    \item \textbf{Postcondicions}: El client web rep notificacions sobre canvis en elements compartits.
    \item \textbf{Escenari principal}:
    \begin{enumerate}
        \item Quan un usuari propietari comparteix un element (a la \texttt{UC-13}), el servei \textbf{FileSharing} envia un missatge a RabbitMQ.
        \item \textbf{SyncService} rep aquest missatge.
        \item Per al client web, \textbf{SyncService} envia una ordre genèrica de refresc (\texttt{updated\_tree}) a l'usuari amb qui s'ha compartit l'element, indicant-li que ha de refrescar la secció d'arxius compartits.
        \item Per al client d'escriptori (Tauri), aquesta funcionalitat no està implementada. El disseny preveia que \textbf{SyncService} actualitzés un \textit{snapshot} específic per als elements compartits, però no es va dur a terme.
    \end{enumerate}
\end{itemize}
